We were using both micro and macro F1 scores as target metrics.
\subsection{Monolingual}
We trained our network on BERT base embeddings for two experiments: {\tt amazon EN-amazon EN} and {\tt amazon EN-organic}. The main interest was in comparing fine-tuned and not fine-tuned models. \\ 
\begin{figure}[h]
\begin{tikzpicture}

\begin{axis}[
    barplot,
    symbolic x coords={am_f, or_f, empty, am_nf, or_nf},
    xticklabels={amazon EN, organic, amazon EN, organic},
    ]
\addplot[fill=TUMBlau] coordinates {(am_f,45.1) (or_f,61.5) (am_nf,53.1) (or_nf,41.8)};
\addplot[fill=TUMBlauHell] coordinates {(am_f,44) (or_f,61.7) (am_nf,52.7) (or_nf,37.3)};
\scorelegend
\end{axis}

\end{tikzpicture}
\caption{Monolingual {\tt amazon EN-amazon EN} and {\tt amazon EN-organic}: with fine-tuning (left) and without fine-tuning (right).}
\label{monolingual_finetune}
\end{figure}
As you can see in Figure \ref{monolingual_finetune}, training only on the {\bf amazon EN}  yields poor results for the {\bf organic}, while fine-tuning on the {\bf organic} significantly drops the performance on the {\bf amazon EN}. Such effect may occur because comments in {\bf amazon} EN and {\bf organic} have different structure or they've been annotated in slightly different ways. \\
Figure \ref{monolingual_finetune} shows that we should not fine-tune the model if we want to get the best results for {\bf amazon EN}. Thus, from now on experiment {\tt amazon~EN-organic} will always assume that fine-tuning on {\bf organic} was performed; experiment {\tt amazon EN-amazon EN} -- that it was not. \\
\begin{figure}[h]

\begin{tikzpicture}

\begin{axis}[barplot_mono, name=en_en, title=\taskEN]
\addplot[fill=TUMBlau] coordinates {(milnet, 53.1) (swn, 32.9) (vader, 47.3) (nltk, 38.2) (svm, 47.4)};
\addplot[fill=TUMBlauHell] coordinates {(milnet, 52.7) (swn, 27.4) (vader, 34) (nltk, 37) (svm, 47.4)};
\end{axis}

\begin{axis}[
    barplot_mono,
    name=en_org,
    at=(en_en.below south),
    anchor=above north,
    title=\taskORG
    ]
\addplot[fill=TUMBlau] coordinates {(milnet, 61.5) (swn, 38.5) (vader, 48) (nltk, 35.7) (svm, 37.8)};
\addplot[fill=TUMBlauHell] coordinates {(milnet, 61.7) (swn, 37.7) (vader, 44.3) (nltk, 18.8) (svm, 28.3)};
\end{axis}

\begin{axis}[
    barplot_mono,
    name=org_org,
    at=(en_org.below south),
    anchor=above north,
    title=\taskORGORG
    ]
\addplot[fill=TUMBlau] coordinates {(milnet, 60.5) (swn, 38.5) (vader, 48) (nltk, 45) (svm, 45.7)};
\addplot[fill=TUMBlauHell] coordinates {(milnet, 60.6) (swn, 37.7) (vader, 44.3) (nltk, 43.8) (svm, 43.7)};

\scorelegend
\end{axis}

\end{tikzpicture}
\caption{Comparison of MilNet and baselines for the monolingual experiments.}
\label{monolingual_results}
\end{figure}

In Figure \ref{monolingual_results} you can see the results for all the monolingual experiments. The plots show that task {\tt organic-organic} is easier than {\tt amazon~EN-organic} not only for our model but for the baselines as well. Again, it may be the evidence of some notable distinctions between {\bf amazon EN} and {\bf organic} datasets. \\
\subsection{Cross-lingual}
Figure \ref{crosslingual_results} shows the results for our cross-lingual experiments performed on different initial embeddings. We can see that for the first two experiments all the embeddings produced similar results. Interestingly, although {\bf amazon DE} was used only as a test set, all the embeddings yielded better F1-micro scores on the {\tt amazon EN-amazon DE} then on the {\tt amazon EN-amazon DE}. Also, Textblob outperformed both NLTK and SVM as well. Possibly, this result could be explained by the quality of {\bf amazon DE} and the features of the German language --- for example, one could suggest that Germans express their opinions in a clearer way. \\
\begin{figure}[h]
\begin{tikzpicture}

\begin{axis}[
    barplot_en,
    name=en_en,
    ]
\addplot[fill=TUMBlau] coordinates {(bert, 48.1) (roberta, 49.3) (xling, 51.6) (nltk, 38.2) (svm, 47.4)};
\addplot[fill=TUMBlauHell] coordinates {(bert, 48.2) (roberta, 48.9) (xling, 51.2) (nltk, 37) (svm, 47.4)};
\end{axis}


\begin{axis}[
    barplot_en,
    name=en_org,
    at=(en_en.below south),
    anchor=above north,
    xticklabels={BERT, RoBERTa, XLING, NLTK, SVM},
    ]
\addplot[fill=TUMBlau] coordinates {(bert, 56.6) (roberta, 58.1) (xling, 52.1) (nltk, 35.7) (svm, 37.8)};
\addplot[fill=TUMBlauHell] coordinates {(bert, 56) (roberta, 55.7) (xling, 49.8) (nltk, 18.8) (svm, 28.3)};
\end{axis}

\begin{axis}[
    barplot_de,
    name=en_de,
    at=(en_org.below south),
    anchor=above north,
    xticklabels={BERT, RoBERTa, XLING, TextBlob},
    ]
\addplot[fill=TUMBlau] coordinates {(bert, 50.7) (roberta, 53.8) (xling, 66.4) (textblob, 37.1)};
\addplot[fill=TUMBlauHell] coordinates {(bert, 41.5) (roberta, 40.2) (xling, 59.3) (textblob, 36.6)};

\scorelegend
\end{axis}

\end{tikzpicture}
\caption{Comparison of different multilingual embeddings and baselines: {\tt amazon EN-amazon EN} (top), {\tt amazon EN-organic} (middle) and {\tt amazon~EN-amazon~DE} (bottom)}.
\label{crosslingual_results}
\end{figure}
\subsection{Different context level embeddings}
In our next experiment, we compared two models trained on BERT multilingual embeddings with different levels of context: "comment as a context"\ and "sentence as a context". As was described in section \ref{contexLeveLEmbeddings}, for getting valid embeddings we had to discard a large portion of the data. Thus, the results in the Figure \ref{crosslingual_context} for the {\tt amazon EN-amazon EN} are worse than the ones shows in the Figure \ref{crosslingual_results}. \\
Note that comment-level context coincides with sentence-level context for comments having only one sentence. For this reason, we didn't run this experiment for the {\bf organic} dataset. \\
As you can see in Figure \ref{crosslingual_context}, for our task comment-level context didn't cause any significant improvement in the results. \\

\begin{figure}
\begin{tikzpicture}
\begin{axis}[
    barplot,
    symbolic x coords={en_c, de_c, empty, en_s, de_s},
    xtick=data,
    xticklabels={amazon EN, amazon DE, amazon EN, amazon DE},
    ]
\addplot[fill=TUMBlau] coordinates {(en_c,52.7) (de_c,53.1) (en_s,48.6) (de_s,52.4)};
\addplot[fill=TUMBlauHell] coordinates {(en_c,52.6) (de_c,48.9) (en_s,48.2) (de_s,50.4)};
\scorelegend
\end{axis}
	
\end{tikzpicture}

\caption{Cross-lingual {\tt amazon EN-amazon EN} and {\tt amazon EN-amazon DE}: comment-level context (left) and sentence-level context (right).}
\label{crosslingual_context}
\end{figure}
\subsection{Two-class}
As our last step, we trained a model for predicting one of the two sentiments: {\it negative} or {\tt positive}. 
\begin{figure}[h]
\begin{tikzpicture}

\begin{axis}[
    barplot_en_en,
    name=en_en,
    ]
\addplot[fill=TUMBlau] coordinates {(bert, 48.2) (roberta, 48.9) (xling, 51.2) (nltk, 37.0) (svm, 47.4)};
\addplot[fill=TUMBlauHell] coordinates {(bert, 65.1) (roberta, 71.2) (xling, 72.6) (nltk, 54.6) (svm, 65.8)};
\end{axis}

\begin{axis}[
    barplot_en_org,
    name=en_org,
    at=(en_en.below south),
    anchor=above north,
    ]
\addplot[fill=TUMBlau] coordinates {(bert, 56.0) (roberta, 55.7) (xling, 49.8) (nltk, 18.8) (svm, 28.3)};
\addplot[fill=TUMBlauHell] coordinates {(bert, 73.9) (roberta, 75.5) (xling, 83.6) (nltk, 39.5) (svm, 47.3)};
\end{axis}

\begin{axis}[
    barplot_en_de,
    name=en_de,
    at=(en_org.below south),
    anchor=above north,
    ]
\addplot[fill=TUMBlau] coordinates {(bert, 41.5) (roberta, 40.2) (xling, 59.3) (textblob, 36.6)};
\addplot[fill=TUMBlauHell] coordinates {(bert, 69.2) (roberta, 73.0) (xling, 71.5) (textblob, 49.7)};
\legend{3 classes, 2 classes}
\end{axis}

\end{tikzpicture}
\caption{Obtained F1-macro scores for the two-class task.}
\label{two_class_results}
\end{figure}
Figure \ref{two_class_results} shows that for all the experiments reducing the number of classes triggers major improvements in F1-macro score. Similarly to the Figure \ref{crosslingual_results}, for the two-class problem {\tt amazon EN-amazon DE} produces better results than {\tt amazon EN-amazon EN}.